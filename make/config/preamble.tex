%%%%%%%%%%%%%%%%%%%%%%%%%%%%%%%%%%%%%%%%%
%% Definición de colores para usar en el
%% metadata.yaml y en este documento
%%%%%%%%%%%%%%%%%%%%%%%%%%%%%%%%%%%%%%%%%

\definecolor{INLINECODECOLOR1}{HTML}{BDDBE2 } %azul gris
\definecolor{URLCOLOR1}{HTML}{CD5C5C} %rojo salmon
\definecolor{TOCCOLOR1}{HTML}{29282A}
\definecolor{TOCCOLOR2}{HTML}{FF4500} %rojo naranja
\definecolor{INLINECODECOLOR2}{HTML}{FFE4B5} %amarillo Moccasin



%%%%%%%%%%%%%%%%%%%%%%%%%%%%%%%%%%%%%%%%%
%% Resaltado del código en linea
%%%%%%%%%%%%%%%%%%%%%%%%%%%%%%%%%%%%%%%%%

\usepackage[most]{tcolorbox}

%%\colorlet{LightLavender}{Lavender!40!}
\tcbset{on line, 
        boxsep=0pt, left=0.7pt,right=0pt,top=0pt,bottom=0pt,
        colframe=white,colback=INLINECODECOLOR1,  
        highlight math style={enhanced}
        }

% Guardamos la definición original para que nuestra definición no sea recursiva
\let\oldtexttt\texttt
\renewcommand{\texttt}[1]{
    \tcbox{\oldtexttt{#1}} 
}


%%%%%%%%%%%%%%%%%%%%%%%%%%%%%%%%%%%%%%%%%
%% Para generar texto de prueba
%%%%%%%%%%%%%%%%%%%%%%%%%%%%%%%%%%%%%%%%%

\usepackage{lipsum}

%%%%%%%%%%%%%%%%%%%%%%%%%%%%%%%%%%%%%%%%%
%% Configuración para que las líneas
%% de bloques de código se corten y 
%% no ocupen el margen
%%%%%%%%%%%%%%%%%%%%%%%%%%%%%%%%%%%%%%%%%

\usepackage{fvextra}
\DefineVerbatimEnvironment{Highlighting}{Verbatim}{breaklines,commandchars=\\\{\}}
